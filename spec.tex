\documentclass[fleqn,12pt]{article}

\usepackage[english]{babel}
\usepackage{marginnote}
\usepackage[margin=1in]{geometry}
\usepackage{amsmath,amssymb}
\usepackage{array}
\usepackage{enumerate}
\usepackage{color}
\usepackage[usenames,dvipsnames]{xcolor}
\usepackage{tikz}
\usepackage{hyperref}
\usepackage{longtable}
\usepackage{fancyvrb}

\usepackage{fancyhdr}
\pagestyle{fancy}

\fancyhead{}
\fancyhead[L]{Walk with Talk}
\fancyhead[R]{API Specification}
\setlength{\headheight}{27.2pt}

\renewcommand{\headrulewidth}{0.2pt}
\renewcommand{\arraystretch}{1.5}

\newcommand{\nth}[2]{$ \text{#1}^{\text{#2}} $}
\newcommand{\sub}[2]{$ \text{#1}_{\text{#2}} $}

\title{\textsc{Walk With Talk}}
\author{API Specification}
\date{Last Updated: \today}

\definecolor{darkgreen}{rgb}{0,0.5,0}

\newcommand{\g}{\texttt{GET}}
\newcommand{\p}{\texttt{POST}}
\newcommand{\jsnull}{\texttt{null}}
\newcommand{\auth}{\texttt{\textcolor{darkgreen}{@}authorized}}

\begin{document}
\maketitle

\noindent
On error, end-points will return a JSON string describing the error in the
following format:
\begin{Verbatim}[commandchars=\\\{\}]
  "error": "The error message",
  "code":  200,   \textcolor{darkgreen}{// An HTTP error code}
  "field": "The field that caused the error"
\end{Verbatim}
\noindent
If the error is generic, that is, if the error is not restricted to a certain
field (such as a duplicate username), all fields should be highlighted and
the \texttt{"field"} key will contain \jsnull. A full list of HTTP error code
and their uses can be found
\href{https://en.wikipedia.org/wiki/List_of_HTTP_status_codes}{here}.

The indication that a URL is \auth{} means that it will return a 403 (forbidden)
error unless there has previously been a successful login request sent.

\begin{longtable}{@{} >{\ttfamily/}p{1in} | p{0.8in} | p{4.27in}}
\hline
URL & Methods & Description \\
\hline
login & \g, \p &
Logs in a user.
\newline\newline
The \g{} action retrieves current user details. Results are returned in JSON
format with the following fields:
\begin{Verbatim}
  "username",
  "email",
  "sex",
  "age"
\end{Verbatim}
%
The same fields are expected during the \p{} action, in addition to password
data in the \texttt{"password"} field. This is a SHA-256 hash of the data
entered by the user into the form field.
\newline\newline
The submission will return a 401 if the data does not match an existing user,
with a \jsnull{} value for \texttt{"field"}. On success, there will be a code
200, as well as JSON data just like in the \g{} request.
\\
\hline
register & \p &
Registers a new user.
\newline\newline
The \p{} action will expect the same fields as \texttt{/login}, in addition to
a \texttt{"password\_confirm"} field that follows the same characteristics as
\texttt{"password"}.
\newline\newline
The submission will return a 409 if the username already exists, or if the
passwords don't match. On success, there will the exact same value sent to the
client as in the \texttt{/login} \g{} request.
\\
\hline
schedule \newline \auth & \p, \g &
Updates / Retrieves the current user's schedule.
\newline\newline
The \p{} action expects the following fields in the following format:
\begin{Verbatim}
  "username",
  "monday",
  "tuesday",
  "wednesday",
  "thursday",
  "friday",
  "saturday",
  "sunday"
\end{Verbatim}
%
The username is self-explanatory, but the weekday fields require some more
explanation. Each day is broken up into 30-minute discrete chunks. Each one of
these is assigned a single bit, with a 1 indicating availability, and a 0
indicating the opposite. Thus, each of these fields should be a bit-string of
length 48:
\[
  \frac{24 \text{ hour day} \cdot 60 \text{ minutes an hour}}
       {30 \text{ minute chunks}} = 48 \text{ chunks}
\]
So an example schedule (for one day) could be
\begin{Verbatim}[commandchars=\\\{\}]
  \footnotesize "monday": "000000110000011110001000010000011111100000000000"
\end{Verbatim}
%
The \g{} action will return the same thing outlined in the \p{} section,
in JSON format. If the user doesn't have a schedule yet, all 0's will be
substituted for every day.
\\
\hline
location \newline \auth & \p, \g &
Updates / Retrieves the current user's location.
\newline\newline
The \p{} action expects the following fields:
\begin{Verbatim}
  "username",
  "latitude",
  "longitude"
\end{Verbatim}
%
Both of the coordinate fields should be floating-point values. See the \g{}
action below for the return value.
\newline\newline
The \g{} action will return the most-recent location for the given user in
the same format as the \p{} submission above, in JSON format. If no such
location has been given yet, the coordinate fields will contain \jsnull.
\\
\hline
nearby & \g &
Retrieves nearby ``walkers'' with matching schedules.
\newline\newline
% The \g{} action allows for the following \textit{optional} fields:
% \begin{Verbatim}
%   "",
% \end{Verbatim}
The \g{} action takes no parameters, and returns a JSON array of possible users
with the following information:
\begin{Verbatim}[commandchars=\\\{\}]
  "nearby_users": [
    "username_1" : 1.23, \textcolor{darkgreen}{// Distance}
    "username_2" : 4.57
  ]
\end{Verbatim}
No other information is provided about the users in order to ensure their
privacy. Exact location information isn't provided, either, for the same reason.
The front-end will have to make do with distances from the current user.
The users will already be sorted from closest to furthest.
\\
\hline
\end{longtable}

\subsection*{Handling ``Instant'' Scheduling}
The user of the front-end client may want to find a walking buddy immediately,
and there is no support for this directly through the API outlined above. An
acceptable way of doing this would be to send four requests, as follows:
\begin{itemize}
\item \texttt{GET /schedule} and store the results.
\item \texttt{POST /schedule} with only the current time block available
\item \texttt{GET /nearby} to find everyone available right now
\item \texttt{POST /schedule} to restore the original user schedule
\end{itemize}


\end{document}
